\documentclass[a4paper,norsk]{article}
\usepackage[T1]{fontenc}
\usepackage[utf8]{inputenc}

\usepackage[main=norsk,english]{babel}
\usepackage{parskip}
\usepackage{microtype}
\usepackage{csquotes}
\usepackage[hidelinks]{hyperref}

\usepackage[backend=biber,style=authoryear]{biblatex}
\addbibresource{ref.bib}

\title{Argumenterende tekst \par {\large EXPHIL03 22V, Seminargruppe 209\par}}
\author{Martin Mihle Nygaard\footnote{\href{mailto:martimn@ifi.uio.no}{\nolinkurl{martimn@ifi.uio.no}}}}
\date{}

\begin{document}

\maketitle

\section*{\normalsize\sf%
  Drøft Streets argumentasjon for at moralske verdier er sinnsavhengige. Andre
  tekster som kan trekkes inn inkluderer Dawkins' \emph{Det egoistiske genet}
  og Kants \emph{Grunnlegging av moralens metafysikk}. [Antall ord i teksten: 847.]
}

Argumentasjonen til Street kan brytes ned slik \autocite{street}:

\begin{enumerate}
\def\labelenumi{(\arabic{enumi})}
\item
  Opprinnelsen, \emph{genealogien}, til en oppfatning er viktig for ha kunnskap
  om ting.\footnote{Jf. tradisjonell definisjon av kunnskap: en velbegrunnet,
  sann, overbevisning. \autocite[87]{wikforss}}
\item
  Opprinnelsen kan forsterke eller undergrave, avhengig av kontekst.
\item
  Mennesker har noen grunnverdier, \emph{psykologiske trekk}, som
  stammer fra biologisk evolusjon, siden disse promoterer overlevelse.
\item
  For å finne ut om evolusjonsopprinnelsen til verdiene våre forsterker
  eller undergraver oppfatningen vår, setter vi opp følgende dikotomi:

  \begin{enumerate}
  \def\labelenumii{(\roman{enumii})}
  \item
    Ting har verdi enten fordi vi verdsetter dem
    (\emph{sinnsavhengige}),
  \item
    eller fordi de har verdi i seg selv, uavhengig av noens vurdering
    (\emph{sinnsuavhengige}).
  \end{enumerate}
\item
  Hvis verdier er sinnsuavhengige (altså at de kan være sanne uavhengig
  av vår vurdering), forsterker ikke den evolusjonære opprinnelsen til
  oppfatningene våre deres «sannhet», ettersom evolusjon ikke
  nødvendigvis selekterer for sanne oppfatninger.

  \begin{enumerate}
  \def\labelenumii{(\roman{enumii})}
  \item
    Hvis det finnes moralske sannheter som vi ikke er i stand til å
    fatte, kan vi ikke vite om vi lever riktig. Fra (3) og (4ii).
  \item
    Hvis verdier har verdi fordi vi bestemmer at de har det, er ikke
    opprinnelsen til verdiene problematisk. Fra (3) og (4i).
  \end{enumerate}
\item
  Konklusjonen (5i) er gyldig, men virker så usannsynlig at et av
  premissene bør forkastes. Evolusjonsteorien (3) kan ikke forkastes,
  dermed må \emph{sinnsuavhengige verdier} (4i) være feil.
\end{enumerate}

Jeg er til stor grad enig i konklusjonen, men jeg syntes likevel spesielt (5)
er noe svak. At evolusjon ikke bryr seg om hva som er sant eller usant, betyr
ikke nødvendigvis at vi ikke er i stand til å resonnere oss frem til sanne
påstander. Vi vet for eksempel at tautologier er sanne. Det er veldig
problematisk, egentlig absurd, å betvile vår evne til å vurdere sannhet.
Eksempelvis, kan vi etter samme argumentasjon som i (5) også argumentere for at
evolusjon ikke er sant: vi har ikke utviklet oss til å kjenne igjen sannhet
$\Longrightarrow$ vi kan ikke vite om noe er objektivt sant $\Longrightarrow$
vi kan ikke vite om vi har utviklet oss til å kjenne igjen sannhet \ldots{}
Dette er en motsigelse til første premiss! Vi ender opp med samme problem som
alltid «hvordan kan vi vite om \emph{noe} er sant i det hele tatt?», som blir
litt håpløst å alltid måtte besvare før noe annet kan diskuteres.

Riktig nok, tror jeg at Street snakker om \emph{moralske sannheter}, altså
sannheter om hvordan man skal leve, som kanskje skiller seg fra sannhet som
sådan. Jeg tror nemlig definitivt vi har utviklet oss til å skille sannheter
fra usannheter om virkeligheten; sanne konklusjoner om hvordan virkeligheten
faktisk er er avgjørende for overlevelse. Jeg oppfatter «moralske sannheter»
heller som definisjonsproblem. Vi letter etter et fast fundament å spikre
begrepet «moralsk» til. Jeg tror ikke (4ii) er feil fordi det er uforenelig med
evolusjonsteori, men heller fordi det er håpløst vanskelig å definere begreper,
her moral, uten en kontekst andre veldefinerte begreper; og da har man i
realiteten bare forflyttet problemet. Alternativet, (4ii), for meg er mye mer
overbevisende siden det egentlig bare er et axiom, som også virker intuitivt.
På samme måte sier vi at 1 + 1 = 2 er et axiom, fordi det virker opplagt
selvforklarende, selv om vi ikke klarer å bevise det. At noe har verdi fordi vi
sier at det har verdi er dessuten ganske vanskelig å motbevise.

Kant mener vi kan resonnere oss frem til universelle moralske dommer, som er
uavhengige av noe menneskelig (eller antropologi, som han kaller det). Selv om
Kant skriver lenge før Darwin og hans evolusjonsteori, hevder han faktisk at
«hvis nu naturens egentlige formål for et vesen, som har fornuft og vilje, var
dets \emph{opprettholdelse}, at det skulle \emph{gå det vel}, med et ord dets
\emph{lykke}, så ville å en ha innrettet seg meget dårlig hvis den hadde utsett
seg denne skapnings fornuft til å virkeliggjøre denne hensikt.»
\autocite[4:395]{kant} Han mener at \emph{instinkter} er mye mer effektive for
overlevelse. Nøyaktig \emph{hva} er fornuften vår egentlig godt for? Selv om
dette lukter litt teleologi, har han et poeng i at vår bevissthet og
rasjonalitet kan virke litt overflødig for ren overlevelse. Det kan være at
rasjonalitet og bevissthet bare er en god overlevelsesstrategi; spesielt
rasjonalitet virker nyttig for å tilpasse oss miljøet, men \emph{bevissthet}
virker---for å bruke et fagbegrep---\emph{ganske dust} (i min mening). Det
finnes i dag ingen konsensus om hvorfor bevissthet eksisterer
\autocite{sep-consciousness}. Om det er noe naturen har snublet over, eller er
en konsekvens av en komplisert modellering av vår plass i omgivelsene våre
\autocite[kap. 4]{dawkins}, er det tilsynelatende noe helt spesielt, og jeg finner det
ikke umulig at moralske verdier kan ha sitt utspring her. Jeg kan for eksempel
forestille meg at jeg kan dele moralske vurderinger med andre bevisste vesener,
uavhengig av biologi (eller mangel på biologi: kunstig intelligens).

En mulig kritikk av en forestilling om delte moralske verdier med andre
bevisste vesener, er at vi simpelthen menneskeliggjør alt som har bevissthet.
De eneste tingene vi kjenner til i universet som har bevissthet er mennesker,
og evolusjonen har selektert for at vi har sympati for mennesker. Med andre
ord, våre moralske dommer over sinnstilstander, som å føle smerte og lidelse,
er egentlig bare en tilpasning for å fungere i en sosial gruppe med andre
vesener som evner å kommunisere sine egne preferanser; dette har alltid vært
andre mennesker.

\printbibliography

\end{document}
