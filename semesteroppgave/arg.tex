\documentclass[a4paper,norsk]{article}
\usepackage[T1]{fontenc}
\usepackage[utf8]{inputenc}

\usepackage[main=norsk,english]{babel}
\usepackage{parskip}
\usepackage{microtype}
\usepackage[colorlinks]{hyperref}
\usepackage[nameinlink]{cleveref}
\usepackage{comment}
\usepackage[hang, symbol]{footmisc}

\usepackage[backend=biber,style=authoryear-ibid]{biblatex}
\addbibresource{ref.bib}
\usepackage{csquotes}
\SetCiteCommand{\parencite}
\renewcommand{\mkccitation}[1]{ #1}

\title{\sf Semesteroppgave \par {\large EXPHIL03 22V\par}}
\author{\sf Kandidatnummer: 4684}
\date{}

\begin{document}

\maketitle

\section*{\normalsize\sf%
  Drøft Streets argumentasjon for at moralske verdier er sinnsavhengige. Andre
  tekster som kan trekkes inn inkluderer Dawkins' \emph{Det egoistiske genet}
  og Kants \emph{Grunnlegging av moralens metafysikk}. [Antall ord i teksten:
  \input{word-count.txt}.]
}

Jeg syntes det er vanskelig å finne åpenbare logiske feil i argumentasjonen til
Street. Under vil jeg hovedsaklig bryte den ned, men jeg nevner også en
kantiansk innvending, og kort undrer meg over normative følger.

Argumentasjonskjeden i teksten kan oppsummeres slik \autocite{street}:

%\labelcrefname{enumi}{punkt}{punkter}
\labelcrefformat{enumi}{#2{\itshape#1.}#3}
\labelcrefformat{enumii}{#2{\itshape#1.}#3}
\crefdefaultlabelformat{#2(#1)#3}

\begin{enumerate}
\def\labelenumi{\itshape\oldstylenums{\arabic{enumi}.}}
\def\labelenumii{\itshape\alph{enumii}.}
\item\label{1}
  Opprinnelsen, \emph{genealogien}, til en oppfatning er viktig for ha kunnskap
  om ting.
\item\label{2}
  Opprinnelsen kan forsterke eller undergrave, avhengig av kontekst.
\item\label{3}
  Mennesker har noen grunnverdier, \emph{psykologiske trekk}, som
  stammer fra biologisk evolusjon, siden disse promoterer overlevelse.
\item\label{4}
  For å finne ut om evolusjonsopprinnelsen til verdiene våre forsterker
  eller undergraver oppfatningen vår, setter vi opp følgende dikotomi:

  \begin{enumerate}
  \item\label{4a}
    Ting har verdi enten fordi vi verdsetter dem
    (\emph{sinnsavhengige}),
  \item\label{4b}
    eller fordi de har verdi i seg selv, uavhengig av noens vurdering
    (\emph{sinnsuavhengige}).
  \end{enumerate}
\item\label{5}
  Hvis verdier er sinnsuavhengige (altså at de kan være sanne uavhengig av vår
  vurdering), forsterker ikke den evolusjonære opprinnelsen til oppfatningene
  våre deres «verdimessige sannhet», ettersom evolusjon ikke nødvendigvis
  selekterer for sanne oppfatninger.

  \begin{enumerate} \item\label{5a} Hvis verdier har verdi fordi vi bestemmer
    at de har det, er ikke opprinnelsen til verdiene problematisk. Fra
    \labelcref{3} og \labelcref{4a} \item\label{5b} Hvis det finnes moralske
    sannheter som vi ikke er i stand til å fatte, kan vi ikke vite om vi lever
riktig. Fra \labelcref{3} og \labelcref{4b} \end{enumerate} \item\label{6}
Konklusjonen i \labelcref{5b} er gyldig, men virker så usannsynlig at et av
premissene bør forkastes. Evolusjonsteorien \labelcref{3} kan ikke forkastes
\autocite[jf.][]{dawkins}, dermed må \emph{sinnsuavhengige verdier}
\labelcref{4b} være feil.  \end{enumerate}

Jeg antar at Street i \labelcref{1}~og \labelcref{2}~tar utgangspunkt i en
tradisjonell definisjon av kunnskap: en velbegrunnet, sann, overbevisning ---
hvis et av disse tre kravene ikke gjelder, har man ikke kunnskap
\autocite[87]{wikforss}. Hvis man argumenterer for at et av disse kravene
\emph{aldri} gjelder, må en regnes som \emph{skeptiker}
\autocite[90]{wikforss}. Det siste er hva Streets argumentasjon i
\labelcref{5} er et forsøk på: å undergrave \emph{evidensen}, eller
\enquote{velbegrunnetheten}, for vår moralkunnskap, og dermed vise at dette
fører til en \enquote{global verdimessig skepsis} \autocite[451]{street}, eller
\emph{moralsk skeptisisme}.

Street forkaster moralsk skeptisisme (konklusjonen \labelcref{5b}) på bakgrunn
av at synet på ingen måte er beskrivende for hvordan vi lever.  Hen beskriver
det som lite plausibelt om vår medføtte intuisjon villeder oss: om vi
aksepterer en \enquote{global verdimessig skepsis} kan vi like gjerne
\textcquote[451]{street}{leve livet ved å skrike hele tiden, ved å slå hjul,
eller [...] andre ting}. Det virker som Street er av en flere som ser dette
synet som så absurd at ethvert argument som leder hit må forkastes
\autocite{sep-skepticism-moral}. Jeg vil påpeke at jeg tror man kan være et
helt ålreit, moralsk (i folkelig betydning), menneske og likevel være en
moralsk skeptiker; du bare simpelthen anerkjenner at du ikke har god evidens
for dine moralske vurderinger, men du kan ha \emph{troa} på deres sannhet
\autocite[for litt mer om dette se][]{sep-skepticism-moral}.

Street tilbyr derimot en utvei i den alternative konklusjonen \labelcref{5a}
Ved å akseptere denne --- at det ikke finnes objektive moralske fakta --- er
man ikke lenger en moralsk skeptiker, man er en moralsk \emph{anti-realist}
\autocite{sep-moral-anti-realism}. Dette er et snedig triks, vi forkaster altså
ikke \emph{alle} moralske sannheter, men bare spesifikt de som blir undergravet
av \labelcref{3}

\subsubsection*{Kants innvending}

Som nevnt i innledningen, syntes jeg det er utfordrende å finne gode
innvendinger; argumentasjonen er god. Jeg skal ikke gjøre et forsøk, men jeg lar
Kant prøve.

Som en rasjonalist, mener Kant vi kan resonnere oss frem til universelle
moralske dommer, som er uavhengige av noe menneskelig (eller antropologi, som
han kaller det). Selv om Kant skriver lenge før Darwin og hans evolusjonslære,
hevder han faktisk at «hvis nu naturens egentlige formål for et vesen, som har
fornuft og vilje, var dets \emph{opprettholdelse}, at det skulle \emph{gå det
vel}, med et ord dets \emph{lykke}, så ville å en ha innrettet seg meget dårlig
hvis den hadde utsett seg denne skapnings fornuft til å virkeliggjøre denne
hensikt.» \autocite[4:395]{kant} Han mener at \emph{instinkter} er mye mer
effektive for lykke og overlevelse. Nøyaktig \emph{hva} er fornuften vår
egentlig godt for? Vår bevissthet og rasjonalitet kan tidvis virke litt
overflødig for ren overlevelse. Han står ved at moralske verdier har sitt
utspring her, i fornuften.

Er dette et godt motargument mot \labelcref{5}? Jeg sier nei. For det første,
og mest relevant, virker rasjonalitet absolutt nyttig for å tilpasse oss
miljøet; for mennesker, selvsagt, men også i begrenset kapasitet hos andre dyr.
For det andre, Kants argument er teleologisk: vi har rasjonalitet \emph{for} å
bruke den til å finne moralske sannheter \autocite[4:429]{kant}. Dette er et
lite overbevisende eksistensgrunnlag; rasjonalitet som biologisk tilpasning
virker mer sannsynlig.

\subsubsection*{Konsekvenser}

Street skriver avslutningsvis at våre mest grunnlegende verdier bør forbli
uberørt av deres genealogiske opphav, det er vårt metaetiske perspektiv som må
endres \autocite[453]{street}. Stemmer dette? Det er utenfor artikkelens
omfang, men jeg undres om ikke Streets konklusjon også har \emph{normative}
følger. Hvis vår biologiske utvikling er opphavet til våre verdier, bør vi da
kanskje, f.eks, leve mer i henhold til den?

\citeauthor{dawkins}, i sin utgreiing om evolusjon, er tydelig på at han ikke
er interessert i en \enquote{moraloppfatning} basert på evolusjon, og advarer
mot å leve etter \enquote{genenes lov om hensynsløs egoisme} \autocite[kap.
1]{dawkins}\footnote{Det er verdt å merke seg at senere kapitler av den
reviderte \emph{The Selfish Gene} diskuterer hvordan altruisme og samarbeid ---
som tradisjonelt har høy moralsk verdi --- er en viktig årsak til menneskets
suksess \autocite[kap. 12]{dawkins}, som er litt dissonant med dette.}. Han
gjør et poeng ut av at det kanskje også er tilfredsstillende å overkomme våre
naturlige tilbøyeligheter. Jeg er til stor grad enig, det er en feilslutning å
tenke noe \emph{må} brukes til det det er `skapt' for.

Jeg klarer likevel ikke la være å tenke at dette fordrer et antroposentrisk
moralsyn. Hvis moralen likevel ikke er objektiv, den kommer fra oss selv,
menneskene, virker det utfordrende å komme med gode argumenter for enten
biosentriske eller universelle verdier. Det sagt, `P har verdi, fordi vi
mennesker sier det' er ikke ekvivalent med `P har verdi for oss, fordi P er
nyttig for oss mennesker' \autocite[338]{næss}, men det er uklart for meg hvor
bevisbyrden ligger.

\begin{comment}
Begrepet \enquote{moral} blir, generelt sett, brukt på to forskjellige måter:
\emph{beskrivende} og \emph{normativ} \autocite{sep-morality-definition}.
Altså, bruker vi moral som en beskrivelse på hvordan vi allerede lever, eller
som et normativt ideal om hvordan vi \emph{bør} leve? 

Hen beholder \labelcref{5a} først og fremst
fordi den \emph{er} beskrivende (nødvendigvis, om vi aksepterer \labelcref{3}).
Hva Street egentlig ikke har tatt stilling til er om det finnes en
\emph{normativ} moral. Finnes det en måte vi \emph{bør} leve på, som vi ikke
nødvendigvis allerede gjør? Jeg føler derfor behovet for å plassere Street i en
empirisk vitenskapelig tradisjon, sammen med antropologer, psykologer og
biologer, som hovedsakelig er ute etter å beskrive menneskelig adferd, uten å
diktere hvordan vi bør leve.
\end{comment}

\begin{comment}

Å stille opp \emph{sinns\-avhengig} mot \emph{sinns\-uavhengig} som en dikotomi
(enten eller) gir mening hvis man leter etter et beskrivende moralsystem, men
fungerer ikke helt hvis man tillater begge deler. Hvorfor kan vi ikke akseptere
en \enquote{global verdimessig skepsis} til normative ideer om hvordan vi bør
leve, men hlhuluiyliuyioyuoi

{\color{red}%
  Lorem ipsum.
}

{\color{red}%
  At noe har verdi fordi vi sier det er så, \emph{fiat}\footnote{Som i
  \href{https://no.wikipedia.org/wiki/Fiat-penger}{fiat-penger}.}, leder kanskje
  til en Stirnersk sopilistisk egoisme. Er dette problematisk?
}

\paragraph{\labelcref{4a} Bør omformuleres til et slags aksiom, eller en
definisjon.}
\end{comment}

\begin{comment}
Jeg oppfatter «moralske sannheter» heller som et definisjonsproblem.
Vi leter etter et fast fundament å spikre verdi begrepet til. Jeg tror ikke
\labelcref{4b} er feil fordi det er uforenelig med evolusjonsteori, men heller
fordi det er håpløst vanskelig å definere begreper, her moral, uten en kontekst
andre veldefinerte begreper; og da har man i realiteten bare forflyttet
problemet. Alternativet, \labelcref{4b}, for meg er mye mer overbevisende siden
det egentlig bare er et axiom, som også virker intuitivt. På samme måte sier vi
at 1 + 1 = 2 er et axiom, fordi det virker opplagt selvforklarende, selv om vi
ikke klarer å bevise det. At noe har verdi fordi vi sier at det har verdi er
dessuten ganske vanskelig å motbevise.
\end{comment}

\begin{comment}
Kant mener vi kan resonnere oss frem til universelle moralske dommer, som er
uavhengige av noe menneskelig (eller antropologi, som han kaller det). Selv om
Kant skriver lenge før Darwin og hans evolusjonsteori, hevder han faktisk at
«hvis nu naturens egentlige formål for et vesen, som har fornuft og vilje, var
dets \emph{opprettholdelse}, at det skulle \emph{gå det vel}, med et ord dets
\emph{lykke}, så ville å en ha innrettet seg meget dårlig hvis den hadde utsett
seg denne skapnings fornuft til å virkeliggjøre denne hensikt.»
\autocite[4:395]{kant} Han mener at \emph{instinkter} er mye mer effektive for
overlevelse. Nøyaktig \emph{hva} er fornuften vår egentlig godt for? Selv om
dette lukter litt teleologi, har han et poeng i at vår bevissthet og
rasjonalitet kan virke litt overflødig for ren overlevelse. Det kan være at
rasjonalitet og bevissthet bare er en god overlevelsesstrategi; spesielt
rasjonalitet virker nyttig for å tilpasse oss miljøet, men \emph{bevissthet}
virker ... mer som en byrde heller enn et gode? Det finnes i dag ingen
konsensus om hvorfor bevissthet eksisterer \autocite{sep-consciousness}. Om det
er noe naturen har snublet over, eller er en konsekvens av en komplisert
modellering av vår plass i omgivelsene våre \autocite[kap. 4]{dawkins}, er det
tilsynelatende noe helt spesielt, og jeg finner det ikke umulig at moralske
verdier kan ha sitt utspring her. Jeg kan for eksempel forestille meg at jeg
kan dele moralske vurderinger med andre rasjonelle og bevisste vesener,
uavhengig av biologi (eller mangel på biologi: kunstig intelligens).

En mulig kritikk av en forestilling om delte moralske verdier med andre
rasjonelle vesener, er at vi simpelthen menneskeliggjør alt som har
rasjonalitet. De eneste tingene vi kjenner til i universet som har evnen til å
filosofere er mennesker, og evolusjonen har selektert for at vi har sympati for
mennesker. Med andre ord, våre moralske dommer over sinnstilstander, som å føle
smerte og lidelse, er egentlig bare en tilpasning for å fungere i en sosial
gruppe med andre vesener som evner å kommunisere sine egne preferanser; dette
har alltid bare vært andre mennesker.
\end{comment}

\printbibliography

\end{document}
