\documentclass[a4paper,norsk]{article}
\usepackage[T1]{fontenc}
\usepackage[utf8]{inputenc}

\usepackage[main=norsk,english]{babel}
\usepackage{parskip}
\usepackage{microtype}
\usepackage{csquotes}
\usepackage[hidelinks]{hyperref}

\usepackage[backend=biber,style=authoryear]{biblatex}
\addbibresource{ref.bib}

\title{Analyserende tekst \par {\large EXPHIL03 22V, Seminargruppe 209\par}}
\author{Martin Mihle Nygaard\footnote{\href{mailto:martimn@ifi.uio.no}{\nolinkurl{martimn@ifi.uio.no}}}}
\date{}

\begin{document}

\maketitle

\section*{\normalsize\sf%
  1. Redegjør for Wikforss’ argumentasjon for at kunnskapsresistens er et
  problem i det moderne samfunnet. \emph{Antall ord i denne teksten: 416.}
}

Her er konklusjonen til Wikforss sitert først, ettersom den er ganske
umiddelbart begripelig: \blockquote[{\cite[98]{wikforss}}]{[E]n grunntanke er
  politisk beslutninger reflekterer hvordan innbyggerne vil at samfunnet skal
  se ut. Da er det sentralt at samfunnsborgerne er velinformert om relevante
  samfunnsfakta og om hvilke midler man kan bruke for å nå forskjellige mål.
  [...] Men for å kunne nå disse målene (uansett hvilke disse er) må man vite
  hvordan virkeligheten ser ut og hvilke midler som ville fungere for å nå
  målene, og det krever en offentlig debatt basert på godt grunnlag og
  rasjonell argumentasjon.}

Hun bruker et underliggende enkelt deduktivt argument \autocite[98]{wikforss},
som kan oppsummeres slik:

\begin{itemize}
  \item[\it $p_1$] Politiske beslutninger taes for å oppnå mål samfunnet setter seg;
  \item[\it $p_2$] Kunnskap hjelper oss ta gode beslutninger;
  \item[$\Rightarrow$] Kunnskap hjelper oss ta gode beslutninger for å oppnå
    mål samfunnet setter seg.
\end{itemize}

Her er det motsatte implisitt: at \emph{mangel} på kunnskap gjør at vi tar
\emph{dårlige} beslutninger, og samfunnet vil derfor ha vansker med å oppnå
sine ønsker.

Det første premisset ($p_1$) blir ikke utdypet i særlig grad utover det
innledende sitatet over. Det er tatt som gitt et demokrati hvor borgernes
stemmer og meninger faktisk blir hørt. Hun trekker frem flere eksempler fra
amerikansk politikk; hvordan samfunnsordenen speiler oppfatningene --- og
kunnskapsløsheten --- til folk flest, er en rød tråd gjennom hele teksten.

Andre premiss ($p_2$) virker innlysende: at kunnskap har
\textquote[{\cite[97]{wikforss}}]{stor \emph{instrumentell} verdi, dvs. den
hjelper oss til å nå våre mål}, og blir illustrert med eksempler fra
dagliglivet og hvordan mennesker bruker kunnskap for overlevelse. Men Wikforss
bruker likevel mye krefter på å presiserer hva hun mener med
\enquote{kunnskap} og hva som skiller det fra rene overbevisninger. Wikforss'
definisjon av kunnskap er ofte kjent som \enquote{justified true belief}, og er
en tradisjonell definisjon av kunnskap
\autocites[87]{wikforss}{sep-knowledge-analysis}:\footnote{Her i min
oversettelse av \cite{sep-knowledge-analysis}.}

\blockquote{
  $S$ vet at $p$ hvis, og bare hvis,
  \begin{enumerate}
    \def\labelenumi{\roman{enumi}.}
    \item $p$ er sann;
    \item $S$ tror at $p$;
    \item $S$ har god grunn for å tro at $p$.
  \end{enumerate}
}

Siste punkt kaller Wikforss \emph{evidens} \autocite[87]{wikforss}. Hun
fremhever evidensen som noe særegent menneskelig, og er spesielt viktig for å
overbevise andre om ens vurderinger, og dermed også har stor innvirkning på
vårt samarbeid og vår kommunikasjon \autocite[99]{wikforss}. Det er evidens hun
vektlegger i formuleringen av \enquote{kunnskapsresistens}, og kaller det
\enquote{det store problemet}: at vi er \emph{evidensresistente}, altså at vi
ikke baserer overbevisningene våre på et godt grunnlag \autocite[99]{wikforss}.
Her trekker hun frem psykologiske mekanismer og irrasjonell ønsketenkning som
bakenforliggende årsaker \autocite[99]{wikforss}.

Altså, et moderne samfunn som effektivt oppnår sine mål krever kunnskap.
Kunnskap krever evidens (blant annet). Derfor mener Wikforss det er
\textquote[{\cite[99]{wikforss}}]{urovekkende hvis mennesker blir
\enquote{resistente} mot kunnskap}.

\printbibliography

\end{document}
